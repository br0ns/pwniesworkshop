% xcolor=table is because it seems that beamer uses the xcolor package in a
% strange way and thus don't accept us giving arguments to the package.
\documentclass[slidestop,compress,mathserif, xcolor=table]{beamer}

\usepackage[T1]{fontenc}
\usepackage[utf8]{inputenc}
\usepackage[english]{babel}
\usepackage{listings}
\usepackage{hyperref}

\lstset{
    language=c,                        % choose the language of the code
    basicstyle=\ttfamily\footnotesize, % the size of the fonts that are used for the
    keywordstyle=,
    numbers=left,                      % where to put the line-numbers
    numberstyle=\footnotesize,         % the size of the fonts that are used for the
    numbersep=5pt,                     % how far the line-numbers are from the code
    backgroundcolor=\color{white},     % choose the background color. You must add
    showspaces=false,                  % show spaces adding particular underscores
    showstringspaces=false,            % underline spaces within strings
    showtabs=false,                    % show tabs within strings adding particular
    frame=single,                      % adds a frame around the code
    breaklines=true,                   % sets automatic line breaking
    breakatwhitespace=false            % sets if automatic breaks should only happen at
}

% \usepackage{mdwtab}
% \usepackage{mathenv}
% \usepackage{amsfonts}
% \usepackage{amsmath}
% \usepackage{amssymb}
% \usepackage{amsthm}

\usepackage{semantic}
\renewcommand{\ttdefault}{txtt} % Bedre typewriter font

% Use the NAT theme in uk (also possible in DK)
\usetheme[nat,uk, footstyle=low, TPrawlrimage=pwnie.png]{Frederiksberg}

\definecolor{mypink}{RGB}{255,200,220}
\setbeamercolor{background canvas}{bg=mypink}
%\setbeamercolor{structure}{fg=blue}

% Make overlay sweet nice by having different transparancy depending on how
% "far" ahead the overlay is. AWSOME!!
%\setbeamercovered{highly dynamic}
% possible to shift back, so they are just invisible untill they should overlay
% \setbeamercovered{invisible}

% Extend figures into either left or right margin
% Ex: \begin{narrow}{-1in}{0in} .. \end{narrow} will place 1in into left margin
\newenvironment{narrow}[2]{%
  \begin{list}{}{%
  \setlength{\topsep}{0pt}%
  \setlength{\leftmargin}{#1}%
  \setlength{\rightmargin}{#2}%
  \setlength{\listparindent}{\parindent}%
  \setlength{\itemindent}{\parindent}%
  \setlength{\parsep}{\parskip}}%
\item[]}{\end{list}}



\usepackage{tikz}
\usetikzlibrary{calc,shapes,arrows}
% below is for use of backgrounds (foregrounds are not used so they are not
% added to this specification).
\pgfdeclarelayer{background}
\pgfsetlayers{background,main}


\usepackage{subfigure}


% Write a short text to have that shown in the footer of slides other than the
% title slide.
\title[]{Advanced shellcode}
% A possible subslide.
%\subtitle{Regular-expression based bit coding}


\author[br0ns \and iDolf Hatler]
       {br0ns \and iDolf Hatler}

% Only write DIKU in the footer of slides (except the title slide).
\institute[DIKU]{Datalogisk institut, Københavns universitet}

% Remove the date stamp from the footer of slides (except title slide) by giving
% it no short "text"
\date[]{\today}

\begin{document}

\frame[plain]{\titlepage}

\begin{frame}[c]
    \frametitle{Aftenens Program}

    \begin{itemize}
      \pause
        \item \texttt{nasm} og \texttt{netcat}
      \pause
        \item Trampoliner (``Hvor er jeg?'')
      \pause
        \item NOP-slisker (``Er koden mon her?'')
      \pause
        \item Bindshell og connect-back (``Hvad med over netværket?'')
      \pause
        \item Egg hunters (``Hvor \emph{er} min kode?'')
    \end{itemize}
\end{frame}

\begin{frame}[c]
    \frametitle{Koder, kend din assembler}
    \url{http://www.nasm.us/docs.php} \\

    \begin{itemize}
        \pause\item \texttt{\%define STD\_IN 0 ; std\_in file handle}
        \pause\item \texttt{\%include "my-defines.asm"}
        \pause\item \texttt{nasm nizzle.asm -I defines/linux/}
        \pause
          \begin{itemize}
          \item GOTCHA: afsluttende skråstreg er \textit{vigtig!}
          \end{itemize}
        \pause\item \texttt{nasm nizzle.asm -D DEBUG}
        \pause\lstinputlisting{ifdef.asm}
    \end{itemize}
\end{frame}

\begin{frame}[c]
    \frametitle{Netcat: The TCP/IP Swiss army knife}
    \begin{itemize}
      \item \url{http://nc110.sourceforge.net}
      \item \url{http://www.admon.org/netcat-tcpip-swiss-army-knife}
      \item \texttt{man nc}
    \end{itemize}

    \pause \textbf{OBS:} Der er to udgaver af netcat. netcat-traditional og
    netcat-openbsd. De har forskellige parametre. Eksemplerne bruger
    traditional.
\end{frame}

\begin{frame}[c]
    \frametitle{Mikrolegetime: Netcat}

    \lstset{language=bash, numbers=none}
    \pause
    Forbind til vært på port 23:
    \lstinputlisting{nc-connect.sh}
    \pause
    Lyt på port 1337:
    \lstinputlisting{nc-listen.sh}
    \pause
    Send data:
    \lstinputlisting{nc-send.sh}

    Prøv at sende data til hinanden.
\end{frame}

\begin{frame}[c]
    \frametitle{Hvor er jeg?}

    \pause\lstinputlisting{move-eip.asm}

    \pause Det kan man ikke! \vskip8pt

    \pause Men man kan bruge en trampolin!

    \pause\lstinputlisting{trampolin.asm}

\end{frame}

\begin{frame}[c]
    \frametitle{Legetime 1 - skriv din egen trampolin som udskriver ``Hello
    World!''}

  \pause

    {\scriptsize
      Prøv på at bruge defines (+ includes?). \vskip8pt

      C: \texttt{write(int fd, const void *buf, size\_t counter);
        exit(0)} \vskip8pt

      Assembler: \texttt{eax = 4, ebx = 1 (stdout), ecx = buf, edx =
        count, int 0x80, eax = 1, ebx = 0, int 0x80} \vskip8pt

      \begin{tabular}{|c|p{6cm}|}
        \hline
        \texttt{[BITS 32]} & Skal stå på først linje for få nasm til at bruge
        32-bit \\\hline

        \texttt{mov $a$, $b$} & Flytter $b$ til $a$, hvor $a$ og $b$ er
        registre, hukommelse eller heltal. \\\hline

        \texttt{push r} & Pusher register \texttt{r} \\\hline

        \texttt{pop r} & Pop register \texttt{r} \\\hline

        \texttt{label:} & Laver en label man kan hoppe til (og ikke bruge til
        meget andet her) \\\hline

        \texttt{jmp label}/\texttt{call label} & Hopper/caller en label \\\hline

        \texttt{str1: db "pony",0} & Gemmer den nul-terminerede tekst ``pony''
        i hukommelsen med \texttt{str1} som peger til den \\\hline

      \end{tabular} \vskip8pt

      Kør med: \\\texttt{nasm -f bin evil.asm; ../demo evil}
    }
\end{frame}

\begin{frame}[c]
  \frametitle{NOP-slisker}
  Hvis du ved omtrent hvor din kode er (og kan tåle at lave et par tusinde
  forsøg).
  \lstinputlisting[numbers=none]{nopsled.asm}
  \lstinputlisting[language=python]{nopsled.py}
\end{frame}

\begin{frame}[c]
    \frametitle{Hvad med over netværk?}

    \begin{itemize}
    % \item Internet sockets.
    \item Connect-back.
    \item Bindshell.
    \end{itemize}

    % \pause
    % \vskip10pt

    % \textbf{Mikrolegetime:} Leg lidt med bindshell og connect-back.
    % \texttt{connect-back.asm, bindshell.asm, bindshell.c}

\end{frame}

% Jeg tror ikke vi behøver noget om sockets her -- det forvirrer mere end det
% forklarer

% \begin{frame}[c]
%   \frametitle{Internet sockets}
%   \begin{itemize}
%     \pause
%   \item Styresytemet implementerer en TCP/IP-stak.

%     \pause
%   \item En socket er kendetegnet ved dens
%     \begin{enumerate}
%     \item Lokale port og IP-adresse
%     \item Fjerne port og IP-adresse
%     \item
%     \end{enumerate}
%   \end{itemize}
% \end{frame}

\begin{frame}[c]
  \frametitle{Connect-back}
  Fremgangsmåde:
  \begin{enumerate}
  \item<+-> Åbn en port lokalt og lyt på den
    \lstinputlisting[numbers=none]{nc-listen.sh}
  \item<+-> Kør dit sploit. Exploitet kører en connect-back.
  \item<+-> Connect-backen åbner en forbindelse til den port du lytter på, og
    binder derefter den åbne socket til en shell.
  \item<+-> ...
  \item<+-> Profit!
  \end{enumerate}
\end{frame}

\begin{frame}[c]
  \frametitle{Bindshell}
  Fremgangsmåde:
  \begin{enumerate}
  \item<+-> Kør dit sploit. Exploitet kører en bindshell.
  \item<+-> Bindshellen åbner en port og venter på at nogen forbinder.
  \item<+-> Forbind.
    \lstinputlisting[numbers=none]{nc-connect-bindshell.sh}
  \item<+-> Bindshellen modtager forbindelsen, og binder (deraf ``bind'') den
    åbne socket til en shell.
  \end{enumerate}
\end{frame}

\begin{frame}[c]
  \frametitle{Firewalls}
  Hvad sker der hvis målet ligger bag en firewall? \vskip25pt
  \begin{description}
  \item<+->[Bindshell] Firewallen vil næsten stensikkert blokkere den indgående
    forbindelse fra dig (medmindre du lytter på 1119 eller 3724).
  \item<+->[Connect-back] Det kan være firewallen også filtrerer udgående
    forbindelser, så hvis du har mulighed for det, er det bedst at lytte på
    f.eks. port 80.
  \end{description}
\end{frame}

\begin{frame}
  \frametitle{Mikrolegetime: connect-back og bindshell}

  \pause
  Bindshell:
  {\scriptsize
    \begin{enumerate}
  \pause
    \item Læs (eller ændr -- husk at køre nasm) portnummeret i
      \texttt{bindshell.asm}
  \pause
    \item \texttt{../demo bindshell} (Kaldet hænger)
  \pause
    \item Åbn en ny terminal og forbind tilbage:
      \texttt{nc localhost [port]}
    \end{enumerate}
  }


  \pause
  Connect-back:
  {\scriptsize
    \begin{enumerate}
  \pause
    \item Læs (eller ændr -- husk at køre nasm) portnummeret i
      \texttt{connect-back.asm}. Hvis du vil forbinde tilbage til en anden
      maskine end din egen skal du også ændre IP-adressen.
  \pause
    \item (Terminal A) Lyt på den valgte port: \texttt{nc -l [port]}
  \pause
    \item (Terminal B) \texttt{../demo connect-back}
  \pause
    \item Nu er terminal A forbundet til en shell.
    \end{enumerate}
  }

  \pause
  \vskip15pt
  Prøv også at forbinde til hinanden (hvis I tør).
\end{frame}

\begin{frame}[c]
    \frametitle{Legetime 2 - hack den her server}

    På min maskine kører \texttt{server} på port 30001. Hack den!

\end{frame}

\begin{frame}[c]
    \frametitle{Egg hunters}

    Hvis det er meget begrænset hvilken kode du kan køre, kan du måske bruge en
    egg hunter. Den leder efter ``den rigtige shellcode'' i hukommelsen.

    \vskip15pt

    \pause
    \lstinputlisting{egg-hunter.asm}
\end{frame}

\begin{frame}[c]
  \frametitle{Mikrolegetime: egg hunter}

  Brug \texttt{egg-hunter} til at køre \texttt{huge-shellcode}.

  \vskip15pt

  Hvordan kan du få \texttt{huge-shellcode} ind i programmets hukommelse?
\end{frame}

\begin{frame}[c]
  \frametitle{Bam!}
  \lstinputlisting{egg-hunter-bam.sh}
\end{frame}

\end{document}
