% xcolor=table is because it seems that beamer uses the xcolor package in a
% strange way and thus don't accept us giving arguments to the package.
\documentclass[slidestop,compress,mathserif, xcolor=table]{beamer}

\usepackage[T1]{fontenc}
\usepackage[utf8]{inputenc}
\usepackage[english]{babel}
\usepackage{listings}

\lstset{
    language=c,                        % choose the language of the code
    basicstyle=\ttfamily\footnotesize, % the size of the fonts that are used for the
    keywordstyle=,
    numbers=left,                      % where to put the line-numbers
    numberstyle=\footnotesize,         % the size of the fonts that are used for the
    numbersep=5pt,                     % how far the line-numbers are from the code
    backgroundcolor=\color{white},     % choose the background color. You must add
    showspaces=false,                  % show spaces adding particular underscores
    showstringspaces=false,            % underline spaces within strings
    showtabs=false,                    % show tabs within strings adding particular
    frame=single,                      % adds a frame around the code
    breaklines=true,                   % sets automatic line breaking
    breakatwhitespace=false            % sets if automatic breaks should only happen at
}

% \usepackage{mdwtab}
% \usepackage{mathenv}
% \usepackage{amsfonts}
% \usepackage{amsmath}
% \usepackage{amssymb}
% \usepackage{amsthm}

\usepackage{semantic}
\renewcommand{\ttdefault}{txtt} % Bedre typewriter font

% Use the NAT theme in uk (also possible in DK)
\usetheme[nat,uk, footstyle=low, TPrawlrimage=pony.jpg]{Frederiksberg}

\definecolor{mypink}{RGB}{255,200,220}
\setbeamercolor{background canvas}{bg=mypink}
%\setbeamercolor{structure}{fg=blue}

% Make overlay sweet nice by having different transparancy depending on how
% "far" ahead the overlay is. AWSOME!!
%\setbeamercovered{highly dynamic}
% possible to shift back, so they are just invisible untill they should overlay
% \setbeamercovered{invisible}

% Extend figures into either left or right margin
% Ex: \begin{narrow}{-1in}{0in} .. \end{narrow} will place 1in into left margin
\newenvironment{narrow}[2]{%
  \begin{list}{}{%
  \setlength{\topsep}{0pt}%
  \setlength{\leftmargin}{#1}%
  \setlength{\rightmargin}{#2}%
  \setlength{\listparindent}{\parindent}%
  \setlength{\itemindent}{\parindent}%
  \setlength{\parsep}{\parskip}}%
\item[]}{\end{list}}



\usepackage{tikz}
\usetikzlibrary{calc,shapes,arrows}
% below is for use of backgrounds (foregrounds are not used so they are not
% added to this specification).
\pgfdeclarelayer{background}
\pgfsetlayers{background,main}


\usepackage{subfigure}


% Write a short text to have that shown in the footer of slides other than the
% title slide.
\title[]{Lær at lave shellcode}
% A possible subslide.
%\subtitle{Regular-expression based bit coding}


\author[br0ns \and iDolf Hatler \and TM]
       {br0ns \and iDolf Hatler \and TM}

% Only write DIKU in the footer of slides (except the title slide).
\institute[DIKU]{Datalogisk institut, Københavns universitet}

% Remove the date stamp from the footer of slides (except title slide) by giving
% it no short "text"
\date[]{\today}

\begin{document}

\frame[plain]{\titlepage}

\begin{frame}[c]
    \frametitle{Disclaimer}

    SPØRG FOR FANDEN HVIS DET GÅR FOR STÆRKT!
\end{frame}

\begin{frame}[c]
    \frametitle{Hvad er målet?}

    \begin{itemize}
        \pause\item Kontrol over datamaten.
        \pause\item Eksekvering af vores kode.
        \pause\item Shell.
    \end{itemize}
\end{frame}

\begin{frame}[c]
    \frametitle{\texttt{exploitable.c}}

    \lstinputlisting{exploitable.c}
\end{frame}

\begin{frame}[c]
    \begin{center}
      \Huge Demo
    \end{center}
  \end{frame}

\begin{frame}[c]
    \frametitle{Hvad nu?}

    \begin{itemize}
        \pause\item Den hoppede til adresse \texttt{0x41414141}.
        \pause\item Kan vi bruge det?
        \pause\item Hvordan kommer vi videre?
    \end{itemize}
\end{frame}

\begin{frame}[c]
  \frametitle{Lynkursus i maskinarkitektur}
  \begin{itemize}
    \pause\item Hardware
    \pause\item Instruktionssættet
    \pause\item Styresystemet
  \end{itemize}
\end{frame}

\begin{frame}[c]
    \frametitle{Hardware}
    \begin{itemize}
        \pause\item CPU
        \pause\item Hukommelse
        \pause\item IO
    \end{itemize}
\end{frame}

\begin{frame}[c]
    \frametitle{CPU}
    \begin{itemize}
        \pause
      \item Har registre der bruges til midlertidige beregninger (\texttt{EAX,
          EBX, EBP, ESP, ...})

        \pause
      \item Har en instruktionspeger (\texttt{EIP})

        \pause
      \item Kan regne og gøre mange smarte ting (cache, hukommelsesmaskering,
        etc.)
    \end{itemize}\vskip15pt

    \pause Algoritme:
    \begin{enumerate}
        \pause\item Læs en instruktion fra hukommelsen og flyt \texttt{EIP}
        \pause\item Gør hvad instruktionen siger
        \pause\item Gå til punkt 1
    \end{enumerate} \vskip15pt

    \pause Eksempler: \texttt{mov, add, jmp, call, push}

\end{frame}

\begin{frame}[c]
    \frametitle{Og en pony}
    \includegraphics[width=0.8\textwidth]{pony2}
\end{frame}

\begin{frame}[c]
    \frametitle{Hukommelse}
    \pause Hukommelse på X86 er kompliceret.\vskip10pt

    \pause Long story short:
    \begin{itemize}

      \pause
    \item Fra jeres synsvinkel består hukommelsen af adresser fra
      \texttt{0x00000000} til \texttt{0xffffffff}.

      \pause
    \item Hver adresse indeholder 1 byte, men man tilgår normalt 4 bytes af
      gangen (så \texttt{0x1234} er \texttt{0x1234-0x1237}).

      \pause
    \item X86 er little endian (den ``mindst betydende byte'' bliver gemt
      først).  Tallet \texttt{0xDECAFBAD} består af de fire bytes \texttt{0xAD},
      \texttt{0xFB}, \texttt{0xCA} og \texttt{0xDE} \textit{i den rækkefølge}.

      \pause
    \item Hukommelsen er inddelt i forskellige områder. Hvis man prøver at tilgå
      noget udenfor et af disse områder (\textit{segmenter}) får man en
      segmenteringsfejl.

      \pause
    \item Adresserne er lokale for dit program. Flere processer kan bruge
      \texttt{0x1234} uden at tilgå det samme stykke fysiske hukommelse (som
      endda kan være swap på en disk).
    \end{itemize}
\end{frame}

\begin{frame}[c]
    \frametitle{Hukommelse - fortsat}

    Relevante områder i hukommelsen er:
    \begin{itemize}
        \pause\item Programområdet (\texttt{EIP})
        \pause\item Statisk allokerede variable (BSS)
        \pause\item Dataområdet
        \pause\item Hobområdet (dynamisk lager)
        \pause\item Stakområdet (\texttt{ESP/EBP})
    \end{itemize}
\end{frame}

\begin{frame}[c]
    \frametitle{IO}

    \begin{itemize}
        \pause\item Det findes
        \pause\item Du er ligeglad
    \end{itemize}
\end{frame}

\begin{frame}[c]
    \frametitle{Instruktionssættet}

    Generelt er X86 et clusterfuck, men her er nogle nemme:
    \begin{itemize}
        \pause\item Flyt og regn på data (\texttt{mov, add, xor, shl})
        \pause\item Hop (\texttt{jmp, jne, call, ret})
        \pause\item Stakoperationer (\texttt{push, pop})
        \pause\item \texttt{nop}
    \end{itemize}\vskip15pt

    \pause Og et par funky:
    \begin{itemize}
        \pause
      \item \texttt{bsf} (Bit Scan Forward)

        \pause
      \item \texttt{aesenc} (``Perform One Round of an AES Encryption Flow'')

        \pause
      \item \texttt{f2xm1} (Udregn $2^x-1$ i et floating point -register)
    \end{itemize}
\end{frame}

\begin{frame}[c]
    \frametitle{Legetime 1}

    {\scriptsize

    \pause Mål: kald \texttt{/bin/sh} \vskip5pt

    \pause C: \texttt{execve(char *filename, char *argv[], char
      *envp[])}\vskip5pt

    \pause Aassembler: Sæt \texttt{eax = 11, ebx = filename, ecx = argv, edx =
      envp}. Kald derefter på styresystemet med \texttt{int 0x80}.\vskip5pt

    \pause Nyttige instruktioner:

    \begin{tabular}{|c|p{6cm}|}
      \hline
      \texttt{[BITS 32]} & Skal stå på først linje for få nasm til at bruge
      32-bit \\\hline

      \texttt{global \_start} & Gør entry-point synligt for linkeren\\\hline

      \texttt{\_start:} & Entry-point i programmet \\\hline

      \texttt{mov r1, n} & Flytter tallet $n$ til register \texttt{r1} \\\hline

      \texttt{mov r1, r2} & Flytter indholdet af register \texttt{r2} til
      register \texttt{r1} \\\hline

      \texttt{mov r1, label} & Flytter adressen af \texttt{label} til
      \texttt{r1} \\\hline

      \texttt{int 0x80} & Kald styresystemsfunktioner \\\hline

      \texttt{str1: db "pony",0} & Gemmer den nul-terminerede tekst ``pony'' i
      hukommelsen med \texttt{str1} som peger til den \\\hline

      \texttt{array1: dd l1,l2,l3,0} & Gemmer et array af pointers
      (\texttt{l1,l2,l3}) i hukommelsen med \texttt{array1} som
      peger\\\hline

    \end{tabular} \vskip5pt

    \pause Kør med:\\
    \texttt{nasm -f elf evil.asm; ld -melf\_i386 -o evil evil.o; ./evil}

    }
\end{frame}

\begin{frame}[c]
    \begin{center}
      \Huge Demo
    \end{center}
\end{frame}

\begin{frame}[c]
    \frametitle{Styresystemet}

    \begin{itemize}
        \pause
      \item Du snakker med styresystemet ved hjælp af \texttt{int 0x80}

        \pause
      \item \texttt{EAX} er funktionsnummeret (11 = \texttt{execve}). Vælg andre
        tal for andre
        funktioner\footnote<3->{\url{http://asm.sourceforge.net/syscall.html}}.

        \pause
      \item \texttt{EBX}, \texttt{ECX}, \texttt{EDX}, \texttt{ESI}, \texttt{EDI}
        er argumenter (i den rækkefølge).
    \end{itemize}
\end{frame}

\begin{frame}[c]
    \frametitle{Og en pony mere}
    \includegraphics[width=0.5\textwidth]{pony1}
\end{frame}

\begin{frame}[c]
    \frametitle{Back on track 1}

    \begin{itemize}
      \pause\item Vi har nu oversat vores shellcode som et selvstændigt
      program, der kan køres alene.

      \pause\item Problem: Vi bruger labels.

      \pause\item Man kan ikke bruge (den her slags) labels i shellcode.

      \pause\item Men hvordan får vi fat i vores data uden labels?
    \end{itemize}
\end{frame}

\begin{frame}[c]
    \frametitle{Et muligt svar: Brug stakken}

    \pause Idé: Du har en peger til stakken (\texttt{ESP}) og medmindre der er
    gået noget galt har du lov til at gemme data der. Det gør du ved at
    \texttt{push}'e:\vskip8pt

    \begin{itemize}
        \pause\item Afhænger af størrelsen, indsæt selv joke

        \pause\item \texttt{push byte 0x68         ; pusher
          0x0000\textcolor{red}{00}68}

        \pause\item \texttt{push word 0x732f       ; pusher 0x732f}

        \pause\item \texttt{push dword 0x6e69622f  ; pusher 0x6e69622f}
    \end{itemize} \vskip8pt

    \pause Kortere:
    \begin{itemize}
        \item \texttt{push 0x68732f6e}
        \item \texttt{push 0x69622f2f}
    \end{itemize} \vskip8pt

    \pause Og med en smart syntax:
    \begin{itemize}
        \item \texttt{push \`{}n/sh\`{}}
        \item \texttt{push \`{}//bi\`{}}
    \end{itemize}

    Se desuden \texttt{man ascii}.
\end{frame}

\begin{frame}[c]
    \frametitle{Legetime 2}

    \pause Mål: Lav din shellcode om, så den ikke har nogen labels. \vskip5pt

    \pause Nyttige instruktioner:

    \begin{tabular}{|c|p{6cm}|}
    \hline
    \texttt{[BITS 32]} & Skal stå på først linje for få nasm til at bruge 32-bit
    \\\hline

    \texttt{push n} & Pusher tallet $n$ til stakken som 4 bytes \\\hline

    \texttt{push word n} & Pusher tallet $n$ til stakken som 2 bytes\\\hline

    \texttt{push \`{}abcd\`{}} & Pusher de 4 bytes \texttt{a, b, c} og
    \texttt{d} til stakken \\\hline

    \texttt{mov r1, r2} & Flytter indholdet af register \texttt{r2} til register
    \texttt{r1} \\\hline

    \texttt{mov r1, esp} & Flytter stakpegeren til register \texttt{r1} \\\hline

    \texttt{int 0x80} & Udfører systemkald \\\hline
    \end{tabular} \vskip5pt

    \pause Kør med: \\\texttt{nasm -f bin evil.asm; ../demo evil}
\end{frame}

\begin{frame}[c]
    \begin{center}
      \Huge Demo
    \end{center}
\end{frame}

\begin{frame}[c]
    \frametitle{Back on track 2}

    \begin{itemize}
      \pause\item Nu har vi shellcode som ikke behøver en bestemt kontekst for
      at virke.

      \pause\item Den kan smides ind i et andet program og gøre hvad vi har
      lyst.

      \pause\item Problem: Mange steder hvor vi har mulighed for at lirke
      shellcode ind kræver at den ikke indeholder NULL-tegn.

        \pause\item Hvordan fjerner vi dem?
    \end{itemize}
\end{frame}

\begin{frame}[c]
  \frametitle{Trickz til at undgå NULL-bytes}

    \pause Ingen NULL-bytes i koden? \vskip10pt

    \pause Ofte intet problem, men hvad hvis et register skal være tomt (eller
    næsten tomt)? \vskip10pt

    \pause Der findes masser af tricks der kan bruges i forskellige situationer:

    \pause\lstinputlisting{example.asm}

\end{frame}

\begin{frame}[c]
    \frametitle{Legetime 3}

    Lav jeres shellcode om til ikke at indeholde NULL-tegn.
\end{frame}

\begin{frame}[c]
    \begin{center}
      \Huge Demo
    \end{center}
\end{frame}

\begin{frame}[c]
  \frametitle{Spørgsmål?}
\end{frame}

\end{document}
